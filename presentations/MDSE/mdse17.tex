\documentclass[slidetop,mathserif,red]{beamer}

\usepackage[utf8]{inputenc}
\usepackage[T1]{fontenc}
\newcommand{\specs}{\mathcal{SP}} % satisfaction relation for specifications
\usepackage{amsmath}
\usepackage[british]{babel}
\usepackage{graphicx}
\usepackage{lmodern}
\usepackage{url}
\usepackage{verbatim}
\usepackage[all]{xy}
\usepackage{listings}

\usepackage{fancybox}

\definecolor{dgreen}{rgb}{0.0,0.47,0.0}
\definecolor{maroon}{rgb}{0.8,0.5,0.4}

\usetheme{Singapore}
\usecolortheme{lily}

\graphicspath{{images/}}

\title{A Decade of Software Design and Modeling: A Survey to Uncover Trends of the Practice}

\author[Suresh Kumar Mukhiya]{{Suresh Kumar Mukhiya}}

\institute[Bergen University College]{ Western Norway University of Applied Sciences}

\date[Autumn 2017]{PCS 953 / {DAT 353} \\ Spring 2019\\ Bergen \\ Norway}

\subject{Model Driven Engineering MDE\\ www.dpf.hib.no}

\logo{\includegraphics[height=.05\textwidth]{hvl_logo_engelsk}}


\begin{document}


\begin{frame}
  \titlepage
\end{frame}

\begin{frame}{Outline}
  \tableofcontents[currentsection,currentsubsection]
\end{frame}

\section{Introduction}
\subsection{Software Engineering, SE}

\begin{frame}{Survey conducted on two phases }
\begin{itemize}
	\item with 228 software paractitioner
	\item April-December, 2007
	\item March-November, 2017
\end{itemize} 
\begin{center}
	\includegraphics<1>[scale=0.4]{survey}
\end{center}
\end{frame}

\begin{frame}{Survey Structure}

	\begin{itemize}
		\item Topic 1 - \textbf{Fundamentals} - Software design and what is software model 
		\item Topic 2 -  \textbf{Basic Characteristics of practices} - What medium and methods are used for moedling? 
		\item Topic 3 -  \textbf{Life Cycle} - Activities involved in SDLC  
		\item Topic 4 -  \textbf{Platforms} - Tools, methodologies, platforms used in SDLC 
		\item Topic 5 -  \textbf{Efficacy} - Design and development practices 
		\item Topic 6 -  \textbf{Code VS Model centrism} - Challenges in code-centric vs model centric SD
		\item Topic 7 -  \textbf{Open ended and optinal contact info} 
		\item Topic 8 -  \textbf{Demographics }
	\end{itemize}

\end{frame}

\begin{frame}{Goal of the Survey}
\only<1->{
    \begin{itemize}
    \item Uncover \textbf{trends in the practice} of \texttt{software design} and \textbf{adaptation pattern} of \texttt{modeling language}
    \end{itemize}
}
\end{frame}

\begin{frame}{Goal of the Survey}
\only<1->{
	\begin{itemize}
		\item Uncover \textbf{trends in the practice} of \texttt{\underline{software design}} and \textbf{adaptation pattern} of \texttt{\underline{modeling language}}
	\end{itemize}
}
\end{frame}


%%%%%%%%%%%%%%%%%%%%%%%%%%%%%%%%%%%%%%%%%%%%%%%%%%%%%%%%%%%%%%%%%%%%%%%%%%%%%%%%%%%%%%%%%%%%%%%%%%%

\section{Backgrounds}
\begin{frame}{Software Design}

\begin{itemize}

	\item Topic 6 -  \textbf{Code VS Model centrism} - Challenges in code-centric vs model centric SD
	\item Topic 7 -  \textbf{Open ended and optinal contact info} 
	\item Topic 8 -  \textbf{Demographics }
\end{itemize}

\end{frame}

\begin{frame}{Modeling languages}

\begin{itemize}
	
	\item Topic 6 -  \textbf{Code VS Model centrism} - Challenges in code-centric vs model centric SD
	\item Topic 7 -  \textbf{Open ended and optinal contact info} 
	\item Topic 8 -  \textbf{Demographics }
\end{itemize}

\end{frame}

%%%%%%%%%%%%%%%%%%%%%%%%%%%%%%%%%%%%%%%%%%%%%%%%%%%%%%%%%%%%%%%%%%%%%%%%%%%%%%%%%%%%%%%%%%%%%%%%%%%

\section{Survey Results}

\begin{frame}{Demographics }
\begin{center}
	\includegraphics<1>[scale=0.4]{demographics}
\end{center}
\end{frame}

\begin{frame}{Topic 1: What is a software model?}
   \begin{center}
   	\includegraphics<1>[width=\textwidth]{question1}
   \end{center}
\end{frame}

\begin{frame}{Topic 2: Characterization of Practices 1/4}
        \begin{itemize}
        \item Medium and Methods used for modeling

        \item What models are used for?

        \item Reference Materials
        
         \item Participants daily activities
        \end{itemize}
    
    	 \begin{center}
    		\includegraphics<1>[width=\textwidth]{2_1}
    	\end{center}
\end{frame}

\begin{frame}{Characterization of Practices 2/4}
\begin{center}
	\includegraphics<1>[width=\textwidth]{2_2}
\end{center}
\end{frame}

\begin{frame}{Characterization of Practices 3/4}
\begin{center}
	\includegraphics<1>[width=\textwidth]{2_3}
\end{center}
\end{frame}

\begin{frame}{Characterization of Practices 4/4}
\begin{center}
	\includegraphics<1>[width=\textwidth]{2_4}
\end{center}
\end{frame}


\begin{frame}{Topic 3: Life Cycle - 1/2}
Activivties invovled in various development phases of Software Development Life Cycle (SDLC)
\begin{center}
	\includegraphics<1>[width=\textwidth]{3_1}
\end{center}
\end{frame}

\begin{frame}{Life Cycle 2/2}

\begin{center}
	\includegraphics<1>[width=\textwidth]{3_2}
\end{center}
\end{frame}


\begin{frame}{Topic 4: Platforms}
\begin{center}
	\includegraphics<1>[width=\textwidth]{4_1}
\end{center}
\begin{center}
	\includegraphics<1>[width=\textwidth]{4_2}
\end{center}
\end{frame}


\begin{frame}{Topic 5: Efficacy}
\begin{itemize}
	\item Questions related to suitability of the modeling tools
	\item Participants' perceptions of key characteristics of modeling tools
\end{itemize}
\begin{center}
	\includegraphics<1>[width=\textwidth]{5_1}
\end{center}
\end{frame}



\begin{frame}{Topic 6: Code VS Model centralism - 1/2}
\begin{center}
	\includegraphics<1>[width=\textwidth]{6_1}
\end{center}

\begin{itemize}
	\item Results show it is easier to create a prototypes, modify the system, create a reusable system and explain system to others in the form of model 
	\item It is easier to debug, create effecient software system, create a system as soon as possible in code. 
\end{itemize}

\end{frame}


\begin{frame}{Topic 6: Code VS Model centralism - 2/2}
\begin{center}
	\includegraphics<1>[width=\textwidth]{6_2}
\end{center}
\begin{itemize}
	\item Models become out of date and inconsistent with code 
	\item Models can not be easily exchanged between tools
\end{itemize}
\end{frame}



%%%%%%%%%%%%%%%%%%%%%%%%%%%%%%%%%%%%%%%%%%%%%%%%%%%%%%%%%%%%%%%%%%%%%%


%%%%%%%%%%%%%%%%%%%%%%%%%%%%%%%%%%%%%%%%%%%%%%%%%%%%%%%%%%%%%%%%%%%%%%%%%%%%%%%%%%%%%%%%%%%%%%%%%%%%%%%%%%%%%%%%%%%%%%%%
\section{Analysis}

\begin{frame}{Analysis}
    \begin{itemize}
    \item A meta-model is a model of a modelling language 

    \item Meta-models are used to define modelling languages

    \item E.g. in OO modelling a person is an instance of class

    \item Meta-modelling is used to create Domain Specific Modelling Languages DSMLs, i.e. one create language constructs for important domain concepts, e.g. a student and a teacher is instances of persons
    \end{itemize}
\end{frame}

\begin{frame}[t]{Meta-model example}
  \begin{columns}[T]
    \column{.5\textwidth}

    \column{.5\textwidth}
    
  \end{columns}
  \begin{block}{}
    \only<1>{
    \begin{itemize}
      \item Models: first class entities
    \end{itemize}
    }
    \only<2>{
    \begin{itemize}
      \item Models: specified by means of a modelling language
    \end{itemize}
    }
    \only<3>{
    \begin{itemize}
      \item Modelling language: corresponding meta-model + semi-formal semantics
    \end{itemize}
    }
  \end{block}
\end{frame}



\begin{frame}{OMG Meta-modelling Levels}
    \begin{center}
  \begin{tabular}{|l|p{60mm}|}
    \hline
    \textbf{OMG levels} & \textbf{OMG Standards/examples}\\ \hline
    $M_3$: Meta-meta-model & MOF\\ \hline
    $M_2$: Meta-model & UML language\\ \hline
    $M_1$: Model & A UML model: Class "Person" with attributes "name" and "address"\\ \hline
    $M_0$: Instance & An instance of "Person": "Ola Nordmann" living in "Sotraveien 1, Bergen"\\ \hline
  \end{tabular}
    \end{center}
\end{frame}



\begin{frame}{MOF based modelling languages}
        \begin{description}
        \item [UML System on a Chip]  for microchip/hardware/firmware/software definition

        \item [SoaML] for service-oriented architecture
         
        \item [Business Process Modelling Notation]  (BPMN, together with it's XML form BPML and executable form BPEL) examples of a Process Modelling language
        
         \item [SysML] for modelling large, complex systems of software, hardware, facilities, people and processes

        \item [UPDM] for modelling enterprise architectures

        \item [CWM] for data warehouse
        \end{description}
\end{frame}






%%%%%%%%%%%%%%%%%%%%%%%%%%%%%%%%%%%%%%%%%%%%%%%%%%%%%%%%%%%%%%%%%%%%%%%%%%%%%%%%%%%%%%%%%%%%%%%%%%%%%%%%%

\section{Summary}




\begin{frame}{Benefits of MDE}

    \begin{itemize}
    \item Engineers can reason about the system at different abstraction levels
    \item Platform independent models without concern of implementation details
    \item Less errors and faster development speed by automatic software generation
    \item Software adoption by (automatic) model transformations
    \end{itemize}
\end{frame}


\begin{frame}{Chalenges in MDE}

    \begin{itemize}
    \item Modelling languages need to haver the right abstractions, i.e. one need domain specific modelling languages
    \item Specification of constraints integrated in the meta-modelling approach, i.e. graphical modelling formalisms with well defined semantics
    \item MDE traditionally concerned by software architecture and behaviour, need to have technologies for:
        \begin{itemize}
        \item Model management (version control, meta-model evolution)
        \item Model based security engineering
        \item Model based testing, software dependencies, \ldots
        \end{itemize}
    \end{itemize}
\end{frame}



\begin{frame}{State of the art in MDE}
  \begin{description}
  \item [Modeling] UML or EMF used as modelling language

  \item [Model transformations] Rule based (e.g. Atlas) or ad hoc transformations are used

  \item [Meta modelling] Only tool support for 2 levels of meta-modelling

  \item [Tool support] Eclipse based (EMF, GMF) tools

  \item [Software constraints] Specified in text based language (OCL)
  \end{description}
\end{frame}

\begin{frame}{Links to resources}
    \begin{itemize}
    \item \href{http://www.theenterprisearchitect.eu/archive/2009/01/15/mde---model-driven-engineering----reference-guide }{\textcolor[rgb]{0.00,0.00,1.00}{mde-model-driven-engineering-reference-guide}}

    \item \href{http://www.theenterprisearchitect.eu/archive/2008/03/14/model_driven_engineering}{\textcolor[rgb]{0.00,0.00,1.00}{model-driven-engineering}}

    \item \href{http://www.theenterprisearchitect.eu/archive/2008/01/16/mda-model-driven-architecture-basic-concepts}{\textcolor[rgb]{0.00,0.00,1.00}{mda-model-driven-architecture-basic-concepts}}

    \item \href{http://dpf.hib.no/}{\textcolor[rgb]{0.00,0.00,1.00}{\textcolor[rgb]{0.00,0.00,1.00}{Diagram-Predicate-Framework}}}

    \item \href{http://www.eclipse.org/modeling/}{\textcolor[rgb]{0.00,0.00,1.00}{\textcolor[rgb]{0.00,0.00,1.00}{Eclipse-Modeling-Technologies}}}

  \end{itemize}
\end{frame}


\begin{frame}{Litterateur}
    \begin{itemize}
    \item Conceptual data modelling from a categorical perspective. ter Hofstede, A. H., Lippe, E. and  Frederiks, P. J. M. (1996), The Computer Journal, 39(3), 215-231.
    
    \item View updates in a semantic data modelling paradigm. Johnson, M., Rosebrugh, R. and  Dampney C.N. G.  In: Proceedings of the 12th Australasian database conference. IEEE Computer Society, 2001. p. 29-36

   \item Ontologies: Silver Bullet for Knowledge Management and Electronic Commerce. Fensel D. Berlin: Spring-Verlag

   \item Symbolic graphs for attributed graph constraints. Journal of Symbolic Computation, 46(3), 294-315. Orejas, F. (2011)

	\item Diagram predicate framework: a formal approach to MDE. Rutle, A. (2010)

  \end{itemize}
  \end{frame}



\begin{frame}{More Litterateur}
    \begin{itemize}
  

    \item Domain-Specific Languages, Martin Fowler (With Rebecca Parsons), Addison Wesley
    
    \item Prolog-based infrastructure for RDF: performance and scalability. Wielemaker, J., Schreiber, A.T., Wielinga, B.J.
In: (Ed.), The Semantic Web - Proceedings ISWC'03, Sanibel Island, Florida (pp. 644-658). Springer Verlag

	\item Alloy: a lightweight object modelling notation. Jackson, D. (2002). ACM Transactions on Software Engineering and Methodology (TOSEM), 11(2), 256-290  \cite{b1}
  \end{itemize}
  \end{frame}

\begin{frame}[allowframebreaks]
\frametitle{References}
\bibliographystyle{amsalpha}
\bibliography{./lib.bib}
\end{frame}



%%%%%%%%%%%%%%%%%%%%%%%%%%%%%%%%%%%%%%%%%%%%%%%%

\end{document}
